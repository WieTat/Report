% !TEX encoding = UTF-8 Unicode
% !TEX root =  ../Bachelorarbeit.tex


\chapter{Auswertung} \label{cha:Auswertung}
%\textcolor{CorpColor}{\hline}
\begin{tabularx}{\textwidth}{XXXX}
	Kunde & Objekt & Abteilung & Ansprehperson \tabularnewline
\kundeline & \objekt& \abteilung & \ansprechtsperson \tabularnewline
\end{tabularx}
\par
In den Tabellen sind die Durchschnittswerte gem"ass dem EU-GMP-Leitfaden dargestellt.
\par 
\vspace{1ex}
\raum \qquad \kklasse \qquad \betriebszustand

\begin{tabularx}{\textwidth}{|l|c|c|X|c|c|X|c|c|X|c|}
	\hline
	Prüfung & 
	\rotatebox[origin=c]{90}{Befund} & \rotatebox[origin=c]{90}{Warngrenze (WG)} &
	MP mit WG Verletzung &
	\rotatebox[origin=c]{90}{Eingehalten} & \rotatebox[origin=c]{90}{Aktionsgrenze (AG)} &
	MP mit AG Verletzung &
	\rotatebox[origin=c]{90}{Eingehalten} & \rotatebox[origin=c]{90}{Grenzwert (GW)}	&
	MP mit GW Verletzung & \rotatebox[origin=c]{90}{Eingehalten}\\
	\hline
		Luftprobe & 
	< 1  & $\geq$ 1 & 0 von 2 & \CheckBox[print,width=0.6em,height=0.6em,checked,name=ch1]{}& > 1  & 0 von 2 &  \CheckBox[width=0.6em,height=0.6em,checked,name=ch2]{}&
	1 & 0 von 2 &  \CheckBox[print,width=0.6em,height=0.6em,checked,name=ch3]{} \\
	\hline
		Sedimentationsplatten & 
	< 1  & $\geq$ 1 & 0 von 2 & \CheckBox[print,width=0.6em,height=0.6em,checked,name=ch4]{}& > 1  & 0 von 2 &  \CheckBox[width=0.6em,height=0.6em,checked,name=ch5]{}&
	1 & 0 von 2 &  \CheckBox[width=0.6em,height=0.6em,checked,name=ch6]{} \\
\hline
		Kontaktplatten & 
	< 1  & $\geq$ 1 & 0 von 8 & \CheckBox[print,width=0.6em,height=0.6em,checked,name=ch7]{}& > 1  & 0 von 8 &  \CheckBox[width=0.6em,height=0.6em,checked,name=ch8]{}&
	1 & 0 von 8 &  \CheckBox[print,width=0.6em,height=0.6em,checked,name=ch9]{} \\
	\hline
\end{tabularx}


%RAum 2
\vspace{1ex}
\raum \qquad \kklasse \qquad \betriebszustand

\begin{tabularx}{\textwidth}{|l|c|c|X|c|c|X|c|c|X|c|}
	
	%\multicolumn{2}{|l|}{Q-TEC AG}&\multicolumn{2}{|l|}{Kunde}\\
	\hline
	Luftprobe & 
	< 1  & $\geq$ 1 & 0 von 1 & \CheckBox[print,width=0.6em,height=0.6em,checked,name=ch10]{}& > 1  & 0 von 1 &  \CheckBox[width=0.6em,height=0.6em,checked,name=ch11]{}&
	1 & 0 von 1 &  \CheckBox[print,width=0.6em,height=0.6em,checked,name=ch12]{} \\
	\hline
	Sedimentationsplatten & 
	< 1  & $\geq$ 1 & 0 von 1 & \CheckBox[print,width=0.6em,height=0.6em,checked,name=ch13]{}& > 1  & 0 von 1 &  \CheckBox[width=0.6em,height=0.6em,checked,name=ch14]{}&
	1 & 0 von 1 &  \CheckBox[width=0.6em,height=0.6em,checked,name=ch15]{} \\
	\hline
	Kontaktplatten & 
	< 1  & $\geq$ 1 & 0 von 4 & \CheckBox[print,width=0.6em,height=0.6em,checked,name=ch16]{}& > 1  & 0 von 4 &  \CheckBox[width=0.6em,height=0.6em,checked,name=ch17]{}&
	1 & 0 von 4 &  \CheckBox[print,width=0.6em,height=0.6em,checked,name=ch18]{} \\
	\hline
\end{tabularx}

\newpage
\begin{tabularx}{\textwidth}{XXXX}
	Kunde & Objekt & Abteilung & Ansprehperson \tabularnewline
	\kundeline& \objekt& \abteilung & \ansprechtsperson \tabularnewline
\end{tabularx}
\par
In den Tabellen sind die Durchschnittswerte gem"ass dem EU-GMP-Leitfaden dargestellt.
\par 
%%%--------Raum3
\vspace{1ex}
\raum \qquad \kklasse \qquad \betriebszustand

\begin{tabularx}{\textwidth}{|l|c|c|X|c|c|X|c|c|X|c|}
	\hline
	Prüfung & 
	\rotatebox[origin=c]{90}{Befund} & \rotatebox[origin=c]{90}{Warngrenze (WG)} &
	MP mit WG Verletzung &
	\rotatebox[origin=c]{90}{Eingehalten} & \rotatebox[origin=c]{90}{Aktionsgrenze (AG)} &
	MP mit AG Verletzung &
	\rotatebox[origin=c]{90}{Eingehalten} & \rotatebox[origin=c]{90}{Grenzwert (GW)}	&
	MP mit GW Verletzung & \rotatebox[origin=c]{90}{Eingehalten}\\
		\hline
	Luftprobe & 
	< 1  & $\geq$ 1 & 0 von 1 & \CheckBox[print,width=0.6em,height=0.6em,checked,name=ch19]{}& > 1  & 0 von 1 &  \CheckBox[width=0.6em,height=0.6em,checked,name=ch20]{}&
	1 & 0 von 1 &  \CheckBox[print,width=0.6em,height=0.6em,checked,name=ch21]{} \\
	\hline
	Sedimentationsplatten & 
	< 1  & $\geq$ 1 & 0 von 1 & \CheckBox[print,width=0.6em,height=0.6em,checked,name=ch22]{}& > 1  & 0 von 1 &  \CheckBox[width=0.6em,height=0.6em,checked,name=ch23]{}&
	1 & 0 von 1 &  \CheckBox[width=0.6em,height=0.6em,checked,name=ch24]{} \\
	\hline
	Kontaktplatten & 
	< 1  & $\geq$ 1 & 0 von 3 & \CheckBox[print,width=0.6em,height=0.6em,checked,name=ch25]{}& > 1  & 0 von 3 &  \CheckBox[width=0.6em,height=0.6em,checked,name=ch26]{}&
	1 & 0 von 3 &  \CheckBox[print,width=0.6em,height=0.6em,checked,name=ch27]{} \\
	\hline
\end{tabularx}
\vspace{1ex}

%%%--------Raum4

\vspace{1ex}
\raum \qquad \kklasse \qquad \betriebszustand

\begin{tabularx}{\textwidth}{|l|c|c|X|c|c|X|c|c|X|c|}
	
	%\multicolumn{2}{|l|}{Q-TEC AG}&\multicolumn{2}{|l|}{Kunde}\\
	\hline
	Luftprobe & 
	< 1  & $\geq$ 2 & 0 von 2 & \CheckBox[print,width=0.6em,height=0.6em,checked,name=ch28]{}& > 1  & 0 von 2 &  \CheckBox[width=0.6em,height=0.6em,checked,name=ch29]{}&
	1 & 0 von 2 &  \CheckBox[print,width=0.6em,height=0.6em,checked,name=ch30]{} \\
	\hline
	Sedimentationsplatten & 
	< 1  & $\geq$ 1 & 0 von 2 & \CheckBox[print,width=0.6em,height=0.6em,checked,name=ch31]{}& > 1  & 0 von 2 &  \CheckBox[width=0.6em,height=0.6em,checked,name=ch32]{}&
	1 & 0 von 2 &  \CheckBox[width=0.6em,height=0.6em,checked,name=ch33]{} \\
	\hline
	Kontaktplatten & 
	< 1  & $\geq$ 1 & 0 von 6 & \CheckBox[print,width=0.6em,height=0.6em,checked,name=ch34]{}& > 1  & 0 von 6 &  \CheckBox[width=0.6em,height=0.6em,checked,name=ch35]{}&
	1 & 0 von 6 &  \CheckBox[print,width=0.6em,height=0.6em,checked,name=ch36]{} \\
	\hline
\end{tabularx}

%%%--------Raum5
\vspace{1ex}
\raum \qquad \kklasse \qquad \betriebszustand

\begin{tabularx}{\textwidth}{|l|c|c|X|c|c|X|c|c|X|c|}
	
	%\multicolumn{2}{|l|}{Q-TEC AG}&\multicolumn{2}{|l|}{Kunde}\\
	\hline
	Luftprobe & 
	< 1  & $\geq$ 1 & 0 von 8 & \CheckBox[print,width=0.6em,height=0.6em,checked,name=ch37]{}& > 1  & 0 von 8 &  \CheckBox[width=0.6em,height=0.6em,checked,name=ch38]{}&
	1 & 0 von 8 &  \CheckBox[print,width=0.6em,height=0.6em,checked,name=ch39]{} \\
	\hline
	Sedimentationsplatten & 
	< 1  & $\geq$ 1 & 0 von 4 & \CheckBox[print,width=0.6em,height=0.6em,checked,name=ch40]{}& > 1  & 0 von 4 &  \CheckBox[width=0.6em,height=0.6em,checked,name=ch41]{}&
	1 & 0 von 4 &  \CheckBox[width=0.6em,height=0.6em,checked,name=ch42]{} \\
	\hline
	Kontaktplatten & 
	< 1  & $\geq$ 1 & 0 von 20 & \CheckBox[print,width=0.6em,height=0.6em,checked,name=ch43]{}& > 1  & 0 von 20 &  \CheckBox[width=0.6em,height=0.6em,checked,name=ch44]{}&
	1 & 0 von 20 &  \CheckBox[print,width=0.6em,height=0.6em,checked,name=ch45]{} \\
	\hline
\end{tabularx}
\vspace{70pt}% soll dymamisch werden


\begin{tabularx}{\textwidth}{p{4.5cm}XXp{4cm}}


	\multicolumn{2}{l}{Q-TEC AG}&	\multicolumn{2}{l}{\small{\kundeline}}\\


	\small{Messungen geführt:} &\underline{\hspace{3cm}} & \multirow{3}{8cm}{\small{Geprüft und Genehmigt:}}& \\


    \small{Erstellt und Geprüft:} &\underline{\hspace{3cm}} & &\qquad \underline{\hspace{3cm}}\\
    \small{Geprüft und Genehmigt:} &\underline{\hspace{3cm}}&&\\
	
	
\end{tabularx}

\chapter{Messmittel und Proben} \label{cha:messmittelPoben}
\section{Proben} \label{sec:Proben}§

\begin{itemize}

	\item [1] \proben \quad {\ChoiceMenu[height=10pt,width=0.5cm, bordercolor={CorpColor},print,combo,default= 55mm]{}{55mm, 90mm}} \\[1.5ex]
	   \TextField[width=3cm, height=14pt,print, bordercolor={CorpColor}]{LOT:} \quad \TextField[width=2cm, height=14pt,print, bordercolor={CorpColor}]{Ablaufdatum: }\\[1.5ex]
	   \probeiso \qquad \beb\\[1.5ex]
	\item [2] \proben \quad {\ChoiceMenu[height=10pt,width=0.5cm, bordercolor={CorpColor},print,combo,default= 55mm]{}{55mm, 90mm}} \\[1.5ex] 
	   \TextField[width=3cm, height=14pt,print, bordercolor={CorpColor}]{LOT:} \quad \TextField[width=2cm, height=14pt,print, bordercolor={CorpColor}]{Ablaufdatum: }\\[1.5ex]
     \probeiso \qquad \beb\\

\end{itemize}

\section{Messmittel} \label{sec:Messmittel}
\begin{itemize}
	
	\item [1] \TextField[width=2cm, height=14pt,print, bordercolor={CorpColor}]{Microbial Air Sampler, Type:} \quad
	          \TextField[width=3cm, height=14pt,print, bordercolor={CorpColor}]{Herrsteller: }\\[1.5ex]
	          \TextField[width=3cm, height=14pt,print, bordercolor={CorpColor}]{Serien Nr.:}\\[1.5ex]
	          \TextField[width=3cm, height=14pt,print, bordercolor={CorpColor}]{Kalibrierzertifikat Nr.:} \\[1.5ex]
	          \TextField[width=2cm, height=14pt,print, bordercolor={CorpColor}]{Kalibrierintervall.: } \quad \TextField[width=2cm, height=14pt,print, bordercolor={CorpColor}]{Kalibrierzertifikat gütlig bis: }\\
	
\end{itemize} 
%caption{Messmittel}
\chapter{Pr"ufung} \label{cha:pruefung}
\section{Laminar Feld 1, Klasse: A.}\label{sec:LF1}

\kundeline \quad  \objekt \quad  \abteilung \quad  \betriebszustand \\[1.5ex]
{\ChoiceMenu[height=12pt,width=0.5cm, bordercolor={CorpColor},print,combo,default= 1]{Ahzahl Personsn im Raum}{1,2,3,4,5,6,7}}~{\ChoiceMenu[height=12pt,width=0.5cm, bordercolor={CorpColor},print,combo,default= 1]{davon}{1,2,3,4,5,6,7}}~Messtechniker.\\[1.5ex]

Luftproben\\
%\begin{table}[]
%\caption{Luftproben}
	\resizebox{\columnwidth}{!}{%
	\begin{tabular}{@{}|c|c|cc|cc|@{}}
		\hline
		                                &                               &                     \multicolumn{2}{c|}{}                      &            \multicolumn{2}{c|}{davon}            \\ %\cmidrule(l){5-6} 
		                                & \multirow{-2}{*}{Probenahme-} & \multicolumn{2}{c|}{\multirow{-2}{*}{aerob   mesophile Keime}} & \multicolumn{1}{c|}{Schimmelpilze} &   Sporen    \\ %\cmidrule(l){2-6} 
		\multirow{-3}{*}{Probenahmeort} &            (Liter)            & \multicolumn{1}{c|}{(KBE/Platte)} &        (KBE/$m^3$)         &  \multicolumn{1}{c|}{(KBE/$m^3$)}  & (KBE/$m^3$) \\ %\midrule
\hline
		             KL1.1              &             1000              & \multicolumn{1}{c|}{\textless 1}  &        \textless 1         &  \multicolumn{1}{c|}{\textless 1}  & \textless 1 \\ %\midrule
\hline
		             KL1.2              &             1000              & \multicolumn{1}{c|}{\textless 1}  &        \textless 1         &  \multicolumn{1}{c|}{\textless 1}  & \textless 1 \\ \hline
	\end{tabular}%
	}
%\end{table

\vspace{0.5cm}

Sedimentationsproben\\[1.5ex]

\resizebox{\columnwidth}{!}{%
	\begin{tabular}{|l|ccc|}
		\hline
		& \multicolumn{1}{c|}{}            & \multicolumn{2}{c|}{davon}                     \\ \cline{3-4} 
		\multirow{-2}{*}{Probenahmeort} & \multicolumn{1}{c|}{\multirow{-2}{*}{aerob mesophile Keime}} & \multicolumn{1}{c|}{Schimmelpilze} & Sporen      \\ \hline
		& (KBE/4Stunden/Platte)            & \multicolumn{1}{c|}{(KBE/$m^3$)}    & (KBE/$m^3$)    \\ \hline
		SE 1.1  & \multicolumn{1}{c|}{\textless 1}                             & \multicolumn{1}{c|}{\textless 1}   & \textless 1 \\ \hline
		SE 1.2                   & \multicolumn{1}{c|}{\textless 1} & \multicolumn{1}{c|}{\textless 1} & \textless 1 \\ \hline
	\end{tabular}%
}

Kontaktplatten\\[1.5ex]

\resizebox{\columnwidth}{!}{%
	\begin{tabular}{|l|l|c|c|}
		\hline
		&
		&
		\begin{tabular}[c]{@{}c@{}}Gesamtkeimzahl   \\      aerob mesophile Keime\end{tabular} &
		davon   Schimmelpilze \\ \cline{3-4} 
		\multirow{-2}{*}{Probenahmeort} &
		\multirow{-2}{*}{Definition   des Probenahmeortes} &
		(KBE/Platte) &
		(KBE/Platte) \\ \hline
		AK 1.1 & Wand (1.2 m)                     & \textless 1 & \textless 1 \\ \hline
		AK 1.2                         & linker Arbeitsbereich (0.9   m)  & \textless 1 & \textless 1 \\ \hline
		AK 1.3                         & zentraler Bereich (0.9 m)        & \textless 1 & \textless 1 \\ \hline
		AK 1.4                         & rechter Arbeitsbereich   (0.9 m) & \textless 1 & \textless 1 \\ \hline
		AK 1.5                         & linker Arbeitsbereich (0.9   m)  & \textless 1 & \textless 1 \\ \hline
		AK 1.6                         & zentraler Bereich (0.9 m)        & \textless 1 & \textless 1 \\ \hline
		AK 1.7                         & rechter Arbeitsbereich   (0.9 m) & \textless 1 & \textless 1 \\ \hline
		AK 1.8                         & Wand (1.2 m)                     & \textless 1 & \textless 1 \\ \hline
	\end{tabular}%
}
\section{Laminar Feld 2, Klasse: A.}\label{sec:LF2}

\kundeline \quad  \objekt \quad  \abteilung \quad  \betriebszustand \\[1.5ex]
{\ChoiceMenu[height=12pt,width=0.5cm, bordercolor={CorpColor},print,combo,default= 1]{Ahzahl Personsn im Raum}{1,2,3,4,5,6,7}}~{\ChoiceMenu[height=12pt,width=0.5cm, bordercolor={CorpColor},print,combo,default= 1]{davon}{1,2,3,4,5,6,7}}~Messtechniker.\\[1.5ex]

Luftproben\\
%\begin{table}[]
%\caption{Luftproben}
\resizebox{\columnwidth}{!}{%
	\begin{tabular}{@{}|c|c|cc|cc|@{}}
		\hline
		&                               &                     \multicolumn{2}{c|}{}                      &            \multicolumn{2}{c|}{davon}            \\ %\cmidrule(l){5-6} 
		& \multirow{-2}{*}{Probenahme-} & \multicolumn{2}{c|}{\multirow{-2}{*}{aerob   mesophile Keime}} & \multicolumn{1}{c|}{Schimmelpilze} &   Sporen    \\ %\cmidrule(l){2-6} 
		\multirow{-3}{*}{Probenahmeort} &            (Liter)            & \multicolumn{1}{c|}{(KBE/Platte)} &        (KBE/$m^3$)         &  \multicolumn{1}{c|}{(KBE/$m^3$)}  & (KBE/$m^3$) \\ %\midrule
		\hline
		KL1.1              &             1000              & \multicolumn{1}{c|}{\textless 1}  &        \textless 1         &  \multicolumn{1}{c|}{\textless 1}  & \textless 1 \\ %\midrule
		\hline
		KL1.2              &             1000              & \multicolumn{1}{c|}{\textless 1}  &        \textless 1         &  \multicolumn{1}{c|}{\textless 1}  & \textless 1 \\ \hline
	\end{tabular}%
}
%\end{table
	
	\vspace{0.5cm}
	
	Sedimentationsproben\\[1.5ex]
	
	\resizebox{\columnwidth}{!}{%
		\begin{tabular}{|l|ccc|}
			\hline
			& \multicolumn{1}{c|}{}            & \multicolumn{2}{c|}{davon}                     \\ \cline{3-4} 
			\multirow{-2}{*}{Probenahmeort} & \multicolumn{1}{c|}{\multirow{-2}{*}{aerob mesophile Keime}} & \multicolumn{1}{c|}{Schimmelpilze} & Sporen      \\ \hline
			& (KBE/4Stunden/Platte)            & \multicolumn{1}{c|}{(KBE/$m^3$)}    & (KBE/$m^3$)    \\ \hline
			SE 1.1  & \multicolumn{1}{c|}{\textless 1}                             & \multicolumn{1}{c|}{\textless 1}   & \textless 1 \\ \hline
			SE 1.2                   & \multicolumn{1}{c|}{\textless 1} & \multicolumn{1}{c|}{\textless 1} & \textless 1 \\ \hline
		\end{tabular}%
	}
	
	Kontaktplatten\\[1.5ex]
	
	\resizebox{\columnwidth}{!}{%
		\begin{tabular}{|l|l|c|c|}
			\hline
			&
			&
			\begin{tabular}[c]{@{}c@{}}Gesamtkeimzahl   \\      aerob mesophile Keime\end{tabular} &
			davon   Schimmelpilze \\ \cline{3-4} 
			\multirow{-2}{*}{Probenahmeort} &
			\multirow{-2}{*}{Definition   des Probenahmeortes} &
			(KBE/Platte) &
			(KBE/Platte) \\ \hline
			AK 1.1 & Wand (1.2 m)                     & \textless 1 & \textless 1 \\ \hline
			AK 1.2                         & linker Arbeitsbereich (0.9   m)  & \textless 1 & \textless 1 \\ \hline
			AK 1.3                         & zentraler Bereich (0.9 m)        & \textless 1 & \textless 1 \\ \hline
			AK 1.4                         & rechter Arbeitsbereich   (0.9 m) & \textless 1 & \textless 1 \\ \hline
			AK 1.5                         & linker Arbeitsbereich (0.9   m)  & \textless 1 & \textless 1 \\ \hline
			AK 1.6                         & zentraler Bereich (0.9 m)        & \textless 1 & \textless 1 \\ \hline
			AK 1.7                         & rechter Arbeitsbereich   (0.9 m) & \textless 1 & \textless 1 \\ \hline
			AK 1.8                         & Wand (1.2 m)                     & \textless 1 & \textless 1 \\ \hline
		\end{tabular}%
	}


%\begin{description}

%	\item[Aktueller Wissensstand:] Der aktuelle Wissensstand beschreibt, auf welchem Wissensniveau sich der Autor im Moment der Aufnahme der Arbeit befand.
	
%	\item[Entwicklungsstand TYPO3:] Dieses Kapitel befasst sich mit dem grunds\"atzlichen Entwicklungsstand von TYPO3 Version 4 und 5 und den mit der Extension-Entwicklung zusammenh\"angenden Frameworks Extbase, Fluid und FLOW3. Es werden grundlegende Eigenschaften der Frameworks und deren Leistungsf\"ahigkeit skizziert.
	
%	\item[Methoden und Herangehensweisen:] Im Kapitel »Methoden und Herangehensweisen« werden die zur Planung verwendeten Methoden erl\"autert. Die grundlegenden Eigenschaften und der Aufbau des Softwareentwicklungsprozesses \FachbegriffSpezialA{Rup}{Rational Unified Process}{»Short for Rational Unified Process, a software development methodology from Rational. Based on UML, RUP organizes the development of software into four phases, each consisting of one or more executable iterations of the software at that stage of development.« \Zitat[Abs.\,1]{webopedia:Rup}}{RUP} werden erkl\"art. Zudem werden die anzufertigenden Dokumente spezifiziert.
	
	%\item[Die Planung des Webradio-Players:] Dieses Kapitel umfasst die Dokumentation der gesamten Planungsphase des Webradio-Players. Hier wird eine \"Ubersicht \"uber die bereits vorhandene L\"osung geschaffen und anschlie{\ss}end die zur Planung erforderlichen Dokumente des RUP angefertigt.
	
%	\item[Die Entwicklung des Webradio-Players:] Dieses Kapitel enth\"alt die Dokumentation der tats\"achlichen Programmierung der Software. Hier werden die Voraussetzungen zur Implementation gekl\"art und der Verlauf der Entwicklung anhand von Beispielen schrittweise abgearbeitet.
	
%	\item[Fazit und kritische Bewertung:] Im Fazit werden die gemachten Erfahrungen und die Ergebnisse der Planung und Entwicklung abschlie{\ss}end zusammengefasst und kritisch bewertet. Zus\"atzlich wird ein kleiner Ausblick auf Erweiterungsm\"oglichkeiten und m\"ogliche Optimierungsschritte unternommen.

%\end{description}


%\section{Wenige Informationen, wenige Quellen \dots} \label{sec:Quellenlage}

Grunds\"atzlich war es schwierig geeignete Quellen zu den Themen rund um die Technologien zu finden, da sich -- wie bereits erw\"ahnt -- beide Frameworks noch in der Entwicklung befinden. Aus diesem Grund wurde \"uberwiegend aus Online-Quellen zitiert.

