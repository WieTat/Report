% !TEX encoding = UTF-8 Unicode
% !TEX root =  ../Bachelorarbeit.tex
%\begin{acronym}[EuGH]
	
	\newacronym{ak}{AK}{Oberflächenkeimzahlbestimmung}
	\newacronym{lu}{LU}{Aktive Luftkeimzahlbestimmung}
	\newacronym{se}{SE}{Sedimentation}
	\newacronym{KBE}{KBE}{Koloniebildende Einheiten}
	\newacronym{EN}{EN}{Europ\"aische Norm}
	\newacronym{VDI}{VDI}{Verein Deutscher Ingenieure}
	%\newacronym{eu}{EU}{Europ\"aische Union}
	%\setabbreviationstyle[eu]{Europ\"aische Union-EU}
	\newacronym{h13h14}{H13/H14}{Filterklasse gem"assig EN 1822-1}
	\newacronym{ISO}{ISO}{International Organization for Standardization}
	\newacronym{OP}{OP}{Operationssaal}
	\newacronym{PIC}{PIC}{Pharmaceutical Inspection Co-Operation}
	\newacronym{sn}{S/N}{Seriennummer}
	\newacronym{SOP}{SOP}{Sandard  Oprerating Procedure (Standard Arbeitsanweisung}
	\newacronym{SWKI}{SWKI}{Schweizerische Verein von Gebäudetechnik-Ingenieuren}
	\newacronym{WG}{WG}{Warngrenze}
	\newacronym{MPs}{MPs}{Messpunkten}
	\newacronym{MP}{MP}{Messpunkt}
	\newacronym{AGR}{AGr}{Aktionsgrenze}
	\newacronym{GW}{GW}{Grenzwert}
	\newacronym{gmp}{GMP}{Good Manufacturinbg Practice, Gute Herstellungspraxis}
	\newglossaryentry{EU}
	{
		name=EU,
		description={Europ\"aische Union}
	}

%\end{acronym}
